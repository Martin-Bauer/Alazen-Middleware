\documentclass[]{article}



%opening
\title{Testfaelle GeneTranslator, IndexController}
\author{}

\begin{document}

\maketitle

\section{IndexController}
\subsection{Testfall 1 - answerQuery - einfache Intervallanfrage}
Eingabe: 1,3,100,200\\
Ausgabe: zum Intervall gehörende Mutationsobjekte 

\subsection{Testfall 2 - answerQuery - komplexere Intervallanfrage}
Eingabe: 1,3,100,200 ; 1,3,150,350\\
Ausgabe: eine Mutationsliste mit den Ergebnissen beider Anfragen

\subsection{Testfall 3 - answerQuery - unvollständige Intervallanfrage}
Eingabe: 1,100,200\\
Ausgabe: Fehlermeldung über unvollständige Anfrage

\subsection{Testfall 4 - answerQuery - leere Anfrage}
Eingabe: [...]\\
Ausgabe: Fehlermeldung über leere Anfrage

\subsection{Testfall 5 - answerQuery - überspezifizierte Intervallanfrage}
Eingabe: 1,3,100,200,300,400\\
Ausgabe: Fehlermeldung über überspezifizierte Anfrage

\subsection{Testfall 6 - buildIndex - erfolgreicher Indexaufbau}
Eingabe: [...] (Datenbank ist erreichbar)
Ausgabe: Erfolgreich gebauter Index

\subsection{Testfall 7 - buildIndex - erfolgloser Indexaufbau}
Eingabe: [...] (Datenbank ist nicht erreichbar)
Ausgabe: Fehlermeldung über erfolglosen Indexaufbau
\newpage
\section{GeneTranslator}
Der GeneTranslator hat 2 Aufgaben. Zum einen soll er während der Indexerstellung mit Inhalt (also Gennamen und zugeörigen Intervallen) befüllt werden, zum anderen soll eine Suche nach Gennamen in ihm möglich sein.

\subsection{Testfall 1 - addGene - Einfügen in Datenstruktur}
Eingabe: Testgen, 350, 500\\
Ausgabe: das Gen sollte in den Baum eingefügt sein und per searchForGene() findbar sein

\subsection{Testfall 2 - addGene - Doppeltes Einfügen in Datenstruktur}
Eingabe: Testgen, 350, 500\\
		 Testgen, 350, 500\\
Ausgabe: das Gen sollte nur einmal in die Datenstruktur eingefügt werden. Ausgabe, dass das Gen bereits in der Struktur vorhanden ist.

\subsection{Testfall 3 - addGene - Aufruf ohne Parameter}
Eingabe: [...]\\
Ausgabe: Fehlermeldung über  parameterlosen Aufruf

\subsection{Testfall 4 - tranlateToIntervall - erfolgreiche Suche}
Eingabe: Testgen (befindet sich bereits in Datenstruktur)\\
Ausgabe: 350, 500

\subsection{Testfall 5 - tranlateToIntervall - erfolglose Suche}
Eingabe: Testgen2 (befindet sich nich in Datenstruktur)\\
Ausgabe: Fehlermeldung über erfolglose Suche

\subsection{Testfall 6 - tranlateToIntervall - Aufruf ohne Parameter}
Eingabe: [...]\\
Ausgabe: Fehlermeldung über Parameterlosen Aufruf

\subsection{Testfall 7 - completeGeneName - erfolgreiche Suche mit einem Ergebnis}
Eingabe: Test (Testgen1 befindet sich bereits in Datenstruktur)\\
Ausgabe: Testgen1

\subsection{Testfall 8 - completeGeneName - erfolgreiche Suche mit mehreren Ergebnissen}
Eingabe: Test (Testgen1 und Testgen2 befinden sich bereits in Datenstruktur)\\
Ausgabe: Testgen1, Testgen2

\subsection{Testfall 9 - completeGeneName - erfolglose Suche}
Eingabe: Test (Es befindet sich kein Gen mit Präfix Test in der Datenstruktur)\\
Ausgabe: Fehlermeldung über erfolglose Suche
\end{document}

